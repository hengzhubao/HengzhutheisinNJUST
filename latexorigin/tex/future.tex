\chapter{总结与展望}
\label{chap:future}
\section{本文工作总结}
本文在激光空泡全过程机制研究需求日益增加的背景下,在对国内外研究脉络的梳理分析的基础上,重点针对激光空泡在不同的力学环境中展开了实验和数值的研究。
传统上对bubble的研究,多分为气泡和空化泡。其中气泡的研究多集中在静态气泡受迫振动,而空化泡的研究则多集中在其在不同边界的动力学研究中。本文则通过更加物理的模拟方式研究了激光致空泡在自由、软、硬边界的脉动,实验和数值研究了线性排列的三个激光空泡的动力学,以及激光空泡受压力波强迫的动力学。

基于OpenFOAM的算法实现求解器能够实现在各种复杂边界条件下的激光空泡的脉动模拟,并能准确的捕捉其动力学过程和流场信息。通过模拟计算的实施,发现气-液界面的多种形变机制,极短相对距离的爆破形式,次短距离的射流机制,和较远距离的凸起机制。而在射流机制中,其形成的王冠射流还因王冠射流的方向可分为王冠散射和王冠包裹射流两种方式。也发现了油-水界面附近的空泡的多种溃灭机制,包括在远距离时的射流机制,近距离时的断裂加射流机制,以及在极近相对距离时的空泡被横向射流击断,但产生射流未击穿空泡的断裂接触式。在固体边界附近,验证了空泡的多种射流机制,讨论了空泡粘连和不粘连在壁面上的射流机制的不同,特别是极近距离时的超声速针状射流的验证。

采用纳秒激光照明的单帧照相法,概率的捕捉了空泡的脉动。并针对空泡的阴影照片,定义了多个参数来讨论空泡的脉动变化。实验获得了空泡的相互作用机制,并给出了初步的解释,认为空泡在阵列中的生存时间被延长,但最大体积减小。泡间的相互作用使空泡在脉动过程中发生位置移动和形成特殊的形变,特别是中间空泡存在多种形变和溃灭形式,包括被拉伸至不同程度和被压缩多种机制。还通过数值模拟的方式发现了实验难以获得准确信息的现象,即中间空泡溃灭的关键在边缘空泡的射流入射和相互撞击。这为理解多组相互作用的空泡阵列奠定了一定基础。

通过简单且廉价的实验装置产生了强压力脉冲,这种实验装置使空泡云的定点定时和定大小的产生成为可能。本文利用这种实验装置和精密的高速摄像系统,捕捉了空泡受压力波影响的动力学过程。这符合当前科研研究的趋势,即简易化和精细化。在研究空泡受压力波影响的过程中,发现产生在压力波不同相位的空泡,以及压力波针对空泡脉动的不同阶段的入射,形成的空泡脉动具有不同的现象。发现了空泡云和单大空泡的最大泡半径共同地受负压控制的机制。而压力波的频率和振幅也极大的影响空泡的动力学。研究还发现了空泡溃灭辐射声波的特殊减弱和加强。这种对空泡半径的加大和减小以及对空泡溃灭辐射声波的增强和减弱,为未来的空泡作用的人工控制提供了物理基础。

文中首次实现了激光的同相多空泡线性阵列的同时产生。也是首次使用文中的简单方法产生压力波与激光空泡相互作用。获得的结果对未来的基础物理研究、应用在军事和生物医药领域都有一定的价值。通过模拟计算对实验给出了更清晰的物理解释,对未来辅助研究更多不易实验实现和实验观察的情景具有极大帮助。

%材料新
%方法新
%观点新



\section{下一步工作展望}

本文中简单的总结了空泡在几种不同的力学环境下的动力学机制。但更复杂和更细致的研究仍然是未来的研究重点。比如在更多空泡的人工控制,空泡、空泡云对冲击波的隔绝,空泡脉动和能量的增强、减弱等主题。

激光的多点击穿,和多点击穿形成的空泡是一个艰难且潜力巨大的研究方向。多点击穿形成的空泡在研究中通常简化的建立一个多起始点,后期融合的多空泡模型。也有研究将之简单的化为一个内部均一的椭圆的起始点。未来可以利用分束镜获得一组距离极近的击穿点及后续的多个空泡,与长聚焦区域形成的空泡,对比的研究激光的多点击穿现象,及后续多长空泡机制。

在多空泡击穿过程中,形成了多组击穿冲击波。这些冲击波射到空泡上形成反射。这种冲击波过程对空泡能量的释放有怎样的影响尚不清晰。在这个过程中,还存在空泡对冲击波的屏蔽作用,以及空泡对空泡射流方式的改变。未来针对这些现象,将之利用到舰船防护的研究还值得做出努力。

在实验中,我们发现激光聚焦区域的上游方向,因电场强度的不足,没有形成击穿,但在反射的冲击波经过该地以后,便出现大量的空泡云。这个现象没有统一的精确地解释,CD-Ohl等人倾向于认为,水中自然的存在纳米级的空泡,激光对当地空泡进行了一定的加热,在舒张波的作用下被拉伸到光学极限以上。但这个过程中是否存在激光导致得随机相变并不清楚。未来可以利用冲击波和激光正交的衰减方向(几何衰减$\times$高斯衰减)形成特殊的空泡云中空泡的概率分布,更清晰地解释其中的机理。

激光与水的相互作用的研究,出现了阶段性的分裂,有的人关注飞秒和皮秒尺度的激光击穿过程,而流体领域则直接忽略激光击穿的过程,只认为空泡是一个高温高压气团。中间衔接的部分的理论和实验的解释并不充分。对整个过程的全物理建模(即从微观到介观)十分有必要和价值。

在水中存在多种不同核心时,比如气泡或者固体、液体的微纳粒子,激光在这些核心上的传播,及其后续产生的力的热的现象,仍是一片值得开采的富矿。比如激光引起的局部热点形成气泡表面张力的变化,从而导致气泡的移动。这方面的研究将对湿法沉积和生物组织的给药提供一种新的解决方案。

更宏观地,针对激光击穿空泡在复杂环境中的运动,如绪论中的每个方向都有很大的发展潜力。未来,令空泡这种现象为人所控制的脉动,辐射、反射、散射声波,形成预料到的射流等利用空泡的研究必将迎来更多的关注。

