\begin{abstract}

空化气泡是一种在现实中广泛存在,并具有极高应用价值的物理现象。利用激光与水相互作用产生空泡是当前最简单的非介入式空泡产生方式。空泡的存在环境多种多样,本文针对空泡所处的复杂力学环境,实验和数值的研究了空泡在自由、软、硬界面附近的动力学,在多空泡阵列中的动力学,以及空泡与压力波的相互作用动力学。

首先,基于OpenFoam编写了求解Navier-Stokes的求解器,并利用该求解器模拟计算了空泡在真实物性的自由界面附近、水-油界面附近、和液-固界面附近的运动。研究发现空泡在界面附近的行为依赖于其与界面的距离。在近距离时,空泡的运动较为激烈,能直接对界面形成形状或压力的改造。在中等距离时,射流机制占据空泡与界面相互作用的主导,在其中较小距离具有更加激烈的射流机制。在长距离时,虽然有射流的产生,但空泡对界面的作用主要由声波动控制。不同界面间的射流方式差别可以用开尔文脉冲理论预言。

第二,利用衍射光学元件将高能激光分束并聚焦在水里,研究了同相位等大小的三个以对称线性排列的激光空泡的动力学,并利用数值求解器模拟的方法验证研究了该情景下的空泡运动。研究结果显示,边缘空泡在阵列中的表现近乎于在空泡对的动力学。但中心空泡因为受到来自两边空泡的作用,表现出独特的脉动性质。取决于空泡间距,中间空泡在膨胀时或被拉伸或被压缩。在溃灭时,其在远距离时表现为被拉伸后的长空泡溃灭;在中距离时,其在被拉伸的基础上,受两边空泡的射流影响而形成中间断裂式溃灭;在短距离时,中间空泡受两边空泡的挤压而形成饼状溃灭。这项研究为研究空泡面阵打下了基础。

最后,开发了一种廉价地发生与激光空泡相互作用的压力波的方法,即通过撞击产生压缩波,通过自由界面反射产生舒张波,通过控制撞击加速距离可以改变压力波压强。并用这种方法研究了激光空泡与压力波的相互作用,以及常规重力对空泡动力学的影响。研究发现,空泡产生与压力波引入空泡位置的时间差能够影响空泡形成不同的动力学过程。
当压力波的负压相晚于空泡产生而到达空泡位置,空泡在溃灭后,通常会被舒展波拉扯成一团小空泡团簇,而不是一个大空泡。
但压力波的负压相在空泡溃灭前到达空泡,空泡会被拉扯以停止溃灭并回弹或直接膨胀成一个大空泡。而压力波的正压相对空泡的压缩,通常会加强空泡的溃灭,并产生更强的溃灭冲击声压。同时理论计算了空泡的在压力波作用下的辐射声压。空泡在孤立脉动时的某个阶段开始受压力波影响,会形成声辐射加强。空泡诞生在正压范围内,对辐射声压压强没有显著影响。诞生在负压范围内,其减弱和加强取决于溃灭时刻的声压相位。同时发现了相对低频或者相对高振幅的压力波都会造成空泡最大泡半径的超量膨胀。

本文所做的研究,为理清空泡与环境的相互作用机理做出一定的进步,为潜在的应用提供了现实基础,提出的几种实验方法对后续相关研究提供了依据。本文涉及的内容,对特殊环境内的空泡推动液体运动,对空泡射流对固体物质的损伤,对激光声呐的实现,对爆炸的防护,以及医药的人体组织内递送有应用价值。

\keywords{空泡动力学,多空泡, 冲击波,自由界面,固壁面,水-油界面,OpenFOAM}
\end{abstract}


\begin{englishabstract}

Cavitation bubbles are a widely existing physical phenomenon with extremely high application value in reality. Using laser breaking water down to generate cavitation bubbles is currently the simplest and most popular non-invasive way to generate cavitation bubbles. The environment in which cavitation bubbles exist is quite complex and diverse. This work focuses on the complex mechanical environment in which cavitation bubbles are generated, and experimentally and numerically studied the dynamics of cavitation bubbles near free, soft, and hard interfaces, the dynamics in multiple cavitation bubble arrays, and the interaction dynamics between cavitation bubbles and pressure waves.

Firstly, a solver for solving Navier-Stokes equations was coded based on OpenFoam, and this solver was used to simulate the dynamics of cavitation bubbles near free interfaces, water-oil interfaces, and liquid-solid interfaces with real physical properties. The study found that the behavior of cavitation bubbles near the interfaces depends on their distance from the interface. At close distances, the dynamics of cavitation bubbles is more intense and can directly reshape or remodel the shape or pressure of the interface. At intermediate distances, the jet mechanism dominates the interaction between cavitation bubbles and interfaces, with a more intense jet mechanism at smaller distances. At long distances, though jets are generated, the effect of cavitation bubbles on interfaces is mainly controlled by acoustic waves. The difference in jetting styles between different interface combinations can be predicted by Kelvin impulse theory.

Secondly, by using diffractive optical element (DOE) lens to split high-energy laser beams, and then focused in water using an aplanat, the dynamics of three laser-induced cavitation bubbles of the same phase and size, arranged in a symmetrical linear array were studied, while the more detailed dynamics of cavitation bubbles under this scenario was verified by numerical simulation. The results show that the dynamics of verge bubbles in the array is almost equivalent to the dynamics of a pair of cavitation bubbles. However, due to the influence from both sides, the center bubble exhibits unique oscillation properties. Depending on the distance between cavitation bubbles, the center cavitation bubble can be either stretched or compressed during expansion. When collapse, it exhibits elongated cavitation bubble collapse after being stretched at long distances; at intermediate distances, it forms a center fracture collapse under the influence of jets from both sides after being stretched; at short distances, the center cavitation bubble is squeezed by both sides to form a disk-shaped collapse. This study lays the foundation for studying laser-induced cavitation bubble arrays.

Finally, a low-cost method for generating pressure waves that interact with laser-induced cavitation bubbles was developed, namely by generating compression waves through impact and generating rarefaction waves through reflection from a free interface. The pressure of the pressure wave can be changed by controlling the impact acceleration distance. This method was used to study the interaction between laser-induced cavitation bubbles and pressure waves, as well as the influence of conventional gravity on cavitation bubble dynamics. The study found that the time difference between cavitation bubble generation and the introduction of pressure waves to the cavitation bubble location can affect the formation of different dynamic processes of cavitation bubbles. When the negative pressure phase of the pressure wave arrives at the cavitation bubble location later than the generation of the cavitation bubble, after collapsing, the cavitation bubble is usually pulled into clusters of small cavitation bubbles by rarefaction waves instead of a large cavitation bubble. However, when the negative pressure phase of the pressure wave arrives at the cavitation bubble before it collapses, the cavitation bubble will be pulled to stop collapsing and then to rebound or directly expand into a large cavitation bubble. The positive pressure phase of the pressure wave usually strengthens the collapse of cavitation bubbles and produces stronger collapse shock wave pressure. At the same time, bubble collapse radiation sound pressure under pressure wave loading was theoretically calculated. When an isolating single cavitation bubble begins to be affected by a pressure wave at a certain stage, its sound radiation is enhanced. When a cavitation bubble is seeding within a positive pressure phase, it has no significant effect on radiation sound pressure intensity. When seeded within a negative pressure phase, its weakening and strengthening depend on the pressure wave phase at bubble collapse. It was also found that relatively low-frequency or relatively high-amplitude pressure waves both can cause excessive expansion of maximum bubble radius.

The research done in this work has made some progress in clarifying the interaction mechanism between cavitation bubbles and the environment, also can provide a realistic basis for potential applications. And the several experimental approaches proposed provide a basis for subsequent related research. The content covered in this work has application value for promoting liquid motion in special environments with cavitation bubbles, for damage or erosion to solid materials by cavitation bubble jets, for the realization of laser sonar, for explosion protection, and for delivery within human tissues in medicine.

\englishkeywords{cavitation bubble dynamics,multiple bubbles, shock wave, free surface, solid boundary, liquid-liquid interface, OpenFOAM}
\end{englishabstract}